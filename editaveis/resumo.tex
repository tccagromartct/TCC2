\begin{resumo}
O Agromart é uma solução tecnológica com o objetivo de conectar produtores da agricultura familiar e consumidores de produtos orgânicos, por meio de um aplicativo que possibilita a compra e venda desses produtos. A ideia dessa aplicação surgiu em um Hackathon realizado na FGA no ano de 2020, desde então o seu desenvolvimento foi continuado por alunos da UnB-FGA ao longo de cinco trabalhos de conclusão de curso. A solução atualmente consiste em um aplicativo móvel para os consumidores, um servidor com uma API principal para cada CSA e uma API dicionário que conecta os consumidores ao servidor de cada CSA listada. Por se tratar de um trabalho desenvolvido por várias mãos diferentes ao longo do tempo sem seguir uma metodologia específica para conectar cada um dos trabalhos, a aplicação carece de documentação e uma política clara para orientar o desenvolvimento e dar unidade aos trabalhos desenvolvidos. Neste contexto, o presente trabalho de conclusão de curso em um trabalho de reengenharia, organizou os repositórios, documentou as funcionalidades, e integrou tudo que foi desenvolvido em um único ponto de convergência, para dar unidade ao projeto Agromart. Como efeito, este buscou viabilizar a publicação do Agromart, e possibilitou que o desenvolvimento de novas funcionalidades futuramente sigam essa mesma unidade.

 \vspace{\onelineskip}
    
 \noindent
 \textbf{Palavras-chave}: Agromart. comunidade que sustenta a agricultura (CSA). agricultura familiar.
\end{resumo}
