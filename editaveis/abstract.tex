\begin{resumo}[Abstract]
 \begin{otherlanguage*}{english}
   Agromart is a technological solution aimed at connecting family farmers and consumers of organic products through an application that facilitates the buying and selling of these products. The idea for this application originated in a Hackathon held at FGA in the year 2020, and since then, its development has been continued by students from UnB-FGA over the course of four graduation projects. The current solution consists of a mobile app for consumers, a server for each Community Supported Agriculture (CSA), and a dictionary API that connects consumers to the server of each listed CSA. Due to being a project developed by various hands over time without following a specific methodology to connect each of the projects, the application lacks documentation and a clear policy to guide its development and provide unity to the work done. In this context, the present graduation project organized the repositories, documented the functionalities, and integrated the assets developed into a single point of convergence to bring unity to the Agromart project. As a result, this reorganization enabled the publication of Agromart and turned easier the development of new functionalities in the future following the same unity.

   \vspace{\onelineskip}
 
   \noindent 
   \textbf{Key-words}: Agromart. community-supported agriculture (CSA). family farming.
 \end{otherlanguage*}
\end{resumo}
