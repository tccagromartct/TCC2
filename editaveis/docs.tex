\chapter{Alterações de Documentação}
O objetivo das alterações de documentação foi facilitar a vida dos desenvolvedores que contribuirão com o projeto no futuro. Facilitando o rápido entendimento do escopo do sistema e também sua arquitetura, o que é muito relevante visto que o sistema é dividido entre vários repositórios.

\section{API principal e aplicativo móvel}
Nos repositórios da API principal e do aplicativo móvel, foram realizadas atualizações nos arquivos \texttt{README.md}. Essas atualizações incluem imagens do funcionamento de algumas telas do aplicativo, além de breves descrições de todas as funcionalidades implementadas. Na API principal também foi atualizado para incluir instruções de como acessar o painel do administrador. 

\section{Repositório de documentação}
No repositório de documentação, o arquivo \texttt{README.md} também sofreu atualizações, onde foi adicionada uma explicação detalhada sobre a arquitetura do Agromart, com um diagrama que descreve os principais componentes e como eles se relacionam, facilitando o entendimento da API dicionário e como ela faz o roteamento de múltiplas APIs principais que implementam diferentes CSAs, já que não é uma arquitetura muito usual. Dessa forma, a curva de aprendizado do sistema para novos desenvolvedores é menor.

