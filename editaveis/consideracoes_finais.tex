\chapter[Considerações Finais]{Considerações Finais}

O desenvolvimento do Agromart, iniciado em 2020 durante um \textit{Hackathon} na UnB-FGA, representa um esforço contínuo de alunos e professores em busca de soluções tecnológicas para a agricultura familiar. A proposta original do sistema evoluiu ao longo de cinco trabalhos de conclusão de curso, abrangendo aspectos como arquitetura, integração de pagamentos, hospedagem e funcionalidades específicas para as CSAs.

No entanto, ao longo desse processo, a falta de integração desses trabalhos, bem como a ausência de manutenções e de constantes de atualizações de ambientes no sistema, uma vez que os trabalhos de conclusão de curso são periódicos, comprometeu a integridade do projeto. As múltiplas \textit{branches} nos repositórios, sem uma integração e testes efetivos, geraram incompatibilidades entre funcionalidades e novas arquiteturas prejudicando o funcionamento do sistema.

Diante desse cenário, os objetivos deste trabalho foram claros: analisar e unificar as \textit{branches} existentes nos repositórios, realizar testes para identificar e corrigir inconsistências e todas as falhas que impediam o uso e publicação do MVP, efetuar a publicação desse MVP na Play Store e realizar mudanças na documentação. A busca pela unidade do projeto visa não apenas a viabilização da distribuição do aplicativo e do servidor, mas também facilitar o ambiente para receber contribuições externas, fortalecendo o caráter \textit{open-source} do projeto.

A análise e correção de inconsistências é essencial para garantir o funcionamento adequado do Agromart. Repositórios coesos e integrados, para que realizar uma contribuição \textit{open-source} se torne mais simples, um MVP, com o aplicativo disponível para download na Google Play Store e o servidor apto a ser utilizado por CSAs, representa a concretização dos esforços deste trabalho.

Durante o desenvolvimento do presente trabalho, diversas situações desafiadoras foram encontradas, dentre elas, pode-se destacar a etapa de atualização e troca de algumas bibliotecas utilizadas pelo sistema que estavam depreciadas, respeitando a linha tênue entre abandonar as versões antigas prejudiciais ao sistema, mas sem incorporar novas dependências com diferenças que demandassem alterações críticas no projeto que fossem maiores do que o escopo definido inicialmente. Além disso, mesclar diferentes \textit{branches} de trabalhos desenvolvidos anteriormente em um projeto único, demandou um olhar analítico para entender cada um dos diferentes trabalhos realizados, para enfim realizar a síntese de todos eles em uma ramificação única.

Além disso, apesar de não estar nos planos iniciais, a apresentação do produto desenvolvido para CSA da Florestta foi um passo importante para o desenvolvimento do Agromart, já que funcionalidades cruciais para o funcionamento pleno do sistema em uma implantação comercial foram identificadas. Dessa forma, o Agromart pode ser incrementado a fim de satisfazer esses requisitos futuramente e ser utilizado por CSAs reais. Apesar disso, a sensação de positividade é grande pelo fato do sistema já possuir as funcionalidades básicas para o uso de agricultores individuais.

Para trabalhos futuros, destaca-se a implementação das funcionalidades e melhorias sugeridas pelo administrador da CSA da Florestta durante a apresentação do sitema Agromart, que visam atender às necessidades de uma CSA real. Além disso, a reintegração do meio de pagamento, atualmente desativado por questões técnicas, é um ponto de evolução que abre um leque de possibilidades de futuras funcionalidades de gestão financeira da CSA dentro do Agromart.

Os ótimos resultados alcançados até agora são apenas o começo de um grande ciclo, em que o Agromart poderá se tornar uma ferramenta extremamente útil para a agricultura familiar e pode realmente fazer a diferença para cada agricultor e suas comunidades.
